% !TeX root = ./main.tex
%-------------------------------------
% LaTeX Resume for Software Engineers
% Author : Leslie Cheng
% License : MIT
%-------------------------------------

%-------------------------------------------------- BEGIN HERE --------------------------------------------------

%---------------------------------------------------- HEADER ----------------------------------------------------
\IfFileExists{secrets.tex}{\headertype{\linkedin}{\github}{\website}{\phone}{\email}}{\headertype{\linkedin}{\github}{\website}{}{\email}} % Set the order of items here

%-------------------------------------------------- EDUCATION --------------------------------------------------

\section{}{Education\hspace{480pt}}

  \resumeEntryStart
    \resumeEntryTSDL
      {Software Engineering MASc. - Automotive E/E Architectures}{Sep. 2020 -- May. 2024}
      {McMaster University}{Hamilton, ON}
%   \resumeItemListStart
    %     \resumeItem {\textbf{Thesis:} Synchronizing Real-Time Hardware Control Across TSN Capable Networks} 
    % \resumeItemListEnd
  \resumeEntryEnd
% \vspace{-15pt}
  \resumeEntryStart
    \resumeEntryTSDL
      {Mechatronics Engineering \& Management B.Eng.}{Sep. 2014 -- Apr. 2020}
      {McMaster University}{Hamilton, ON}
  \resumeEntryEnd

%-------------------------------------------------- EXPERIENCE --------------------------------------------------

\section{}{Experience\hspace{473pt}}

  \resumeEntryStart
  \resumeEntryTSDL
    {Software Engineer}{Jan. 2023 -- Present}
    {Indie Semiconductor}{Toronto, ON}
  \resumeItemListStart
% \resumeItem {Customized software and hardware solutions to meet specific customer requirements}
      \resumeItem {Developed low-level C drivers supporting image sensors and serializers/deserializers in an RTOS environment}
      \resumeItem {Created tools for extraction, processing, and verification of embedded data from video streams}
      \resumeItem {Engineered host-side tools to facilitate communication with an SoC through I$^2$C and UART}
      \resumeItem {Conducted comprehensive functional safety and timing analyses, ensuring adherence to industry standards and regulatory requirements}
  \resumeItemListEnd
  \resumeEntryEnd

  \resumeEntryStart
  \resumeEntryTSDL
    {Researcher - Automotive Embedded Systems}{May 2020 -- Dec. 2022}
    {Stellantis (McMaster Automotive Resource Centre)}{Hamilton, ON}
  \resumeItemListStart
      \resumeItem {Brought-up pre-production hardware (NXP S32S \& S32K) to support an electric motor control application, configuring peripherals, clock trees, and pin multiplexing}
      \resumeItem {Implemented a real-time motor control firmware in an RTOS based centralized automotive architecture}
      \resumeItem {Integrated precise time synchronization based on Time Sensitive Networking (TSN) over Automotive Ethernet}
      \resumeItem {Utilized advanced tools such as Lauterbach TRACE32 with JTAG debugging and ETM tracing for thorough hardware and software testing, troubleshooting, and optimization}
      \resumeItem {Conducted comprehensive signal verification and timing analysis for networking and motor control applications, ensuring adherence to OEM performance requirements}
  \resumeItemListEnd
  \resumeEntryEnd

  \resumeEntryStart
    \resumeEntryTSDL
      {Embedded Firmware Specialist}{Oct. 2018 -- May 2020}
      {NEUDOSE}{Hamilton, ON}
      \resumeItemListStart
% \resumeItem {Worked on a project funded by the Canadian Space Agency (CSA) to build and launch a satellite}
      \resumeItem {Engineered STM32-based CAN drivers for satellite On-Board Computer with (CSP) network stack support}
      \resumeItem {Developed mission-critical FreeRTOS-based flight software for the On-Board Computer in C/C++}
% \resumeItem {Contributed to the successful integration of the flight software with the On-Board Computer}
      \resumeItem {Designed a prototype Printed Circuit Board (PCB) using Altium Designer, serving as a crucial component in the testing phase of the flight software}
      \resumeItemListEnd
  \resumeEntryEnd

  \resumeEntryStart
  \resumeEntryTSDL
    {Research Assistant - Model Based Design}{May 2017 -- Apr. 2020}
    {McMaster Centre for Software Certification}{Hamilton, ON}
    \resumeItemListStart
    \resumeItem {Developed model-based Pacemaker following Boston Scientific specs using MATLAB Simulink on FRDM-K64F}
    \resumeItem {Implemented real-time Pacemaker configuration and monitoring over UART in MATLAB Simulink}
    \resumeItem {Automated hardware testing over UART utilizing Arm Mbed firmware (C++) and Python}
  \resumeItemListEnd
\resumeEntryEnd
\newpage
% \resumeEntryStart
  %   \resumeEntryTSDL
  %     {Undergraduate Research Assistant (Model Based Software)}{May 2017 -- April. 2020}
  %     {McMaster Centre for Software Certification}{Hamilton, ON}
  %   \resumeItemListStart
  %       \resumeItem {Developed a model based Pacemaker following Boston Scientific's Pacemaker System Specification using MATLAB Simulink running on an embedded target NXP FRDM-K64F}
  %       \resumeItem {Implemented UART communication in MATLAB Simulink configuring and monitoring the Pacemaker in real-time using a python graphical user interface (GUI)} 
  %       \resumeItem {Created an automated hardware testing and verification including Arm Mbed based firmware in C++ and python scripts communicating with the hardware over UART}
  %   \resumeItemListEnd
  % \resumeEntryEnd

  % \resumeEntryStart
  %   \resumeEntryTSDL
  %     {Instructional Assistant Intern (IAI)}{May 2018 -- April 2019}
  %     {McMaster University}{Hamilton, ON}
  %   \resumeItemListStart
  %   % \resumeItem {Presented Graphics Design (CAD) and Programming (Python) concepts to over 100 students weekly}
  %     \resumeItem {Developed Raspberry Pi I$^2$C drivers for an Inertial Measurement Unit (IMU) and a pulse oximeter}
  %     \resumeItem{Created a python testing tool and a Golang server providing students with feedback on python assignments}
  %     \resumeItem{Setup docker containers to automate the print submission and monitoring process on 3D printers}
  %   \resumeItemListEnd
  % \resumeEntryEnd


%-------------------------------------------------- PROJECTS --------------------------------------------------

\section{}{Projects\hspace{490pt}}

  \resumeEntryStart
  \resumeEntryTD
    {RETINA (Realtime Indoor Navigation Assistant)}{May 2020}
  \resumeItemListStart
    \resumeItem{Developed a Real-time Indoor Navigation Assistant, catering to individuals with visual impairment by leveraging Ultra-Wide Band (UWB) technology, achieving sub-meter precision}
    \resumeItem{Implemented BLE communication between the mobile app and Decawave DW1000 UWB transceivers to retrieve the user's real-time position and heading}
    \resumeItem{Utilized Nominatim for reverse geocoding to enhance location-based services and integrated Valhalla for efficient route generation tailored to indoor environments}
    \resumeItem{Contributed to the accessibility and inclusivity of indoor spaces by developing a system that goes beyond traditional navigation, ensuring a smooth and reliable user experience}
  \resumeItemListEnd
  \resumeEntryEnd

  \resumeEntryStart
    \resumeEntryTD
      {Booky}{Jan. 2018}
    \resumeItemListStart
      \resumeItem {Developed a Cross-Platform mobile app using Flutter available on iOS \& Android, enabling users to find books by taking a picture of the cover}
      \resumeItem {Implemented image search functionality using Google Cloud services, allowing users to explore and discover books of interest effortlessly}
    \resumeItemListEnd
  \resumeEntryEnd

\section{}{Training \& Certification\hspace{405pt}}
\resumeEntryStart
    \resumeEntryTSDL
      {JavaScript Algorithms and Data Structures}{}
      {freeCodeCamp}{}
    \resumeEntryTSDL
      {Advanced MATLAB for Scientific Computing}{}
      {Stanford Online}{}
\resumeEntryEnd

\section{}{Publications\hspace{463pt}}
\resumeEntryStart
    \resumeEntryTD
      {Two Simulink Models with Requirements for a Simple Controller of a Pacemaker Device}{Sep. 2022}
      \resumeItemListStart
        \resumeItem{Accepted at the 9th International Workshop on Applied Verification of Continuous and Hybrid Systems}
      \resumeItemListEnd
      \resumeEntryEnd
%-------------------------------------------------- PROGRAMMING SKILLS --------------------------------------------------
%-------------------------------------------------- PROGRAMMING SKILLS --------------------------------------------------
\section{}{Skills\hspace{500pt}}
\resumeEntryStart
  \resumeEntryS{Programming Languages \hspace{2pt}} {\\C, Python, C++, ARM Assembly, JavaScript, Java, Dart, Verilog, SQL}
  \resumeEntryS{Development Tools \hspace{19pt}} {\\CMake, Ninja, GDB, OpenOCD, Git, Docker, SVN}
  \resumeEntryS{Software Development \hspace{2pt}} {\\MATLAB, Simulink, Altium Designer, Lauterbach TRACE32, STM32CubeMX, Keil $\mu$Vision}
  \resumeEntryS{Hardware Platforms \& Architectures \hspace{2pt}} {\\ARM Cortex-M (STM32F, NXP S32K), ARM Cortex-R (NXP S32S), PowerPC (NXP MPC5), FPGA}
  \resumeEntryS{Communication Protocols \& Technologies \hspace{2pt}} {\\CAN, Automotive Ethernet (TSN), UART, SPI, I$^2$C, MQTT, UDP, TCP/IP}
\resumeEntryEnd
