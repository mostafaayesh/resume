% !TeX root = ./main.tex
%-------------------------------------
% LaTeX Resume for Software Engineers
% Author : Leslie Cheng
% License : MIT
%-------------------------------------

%-------------------------------------------------- BEGIN HERE --------------------------------------------------

%---------------------------------------------------- HEADER ----------------------------------------------------
\headertype{\linkedin}{\github}{\website}{\email}{}{} % Set the order of items here
% \headertype{\linkedin}{\github}{\website}{\phone}{\email}{} % Set the order of items here

%-------------------------------------------------- EDUCATION --------------------------------------------------

\section{}{Education\hspace{480pt}}

  \resumeEntryStart
    \resumeEntryTSDL
      {Software Engineering MASc.}{Sep. 2020 -- Dec. 2023}
      {McMaster University}{Hamilton, ON}
    %   \resumeItemListStart
    %     \resumeItem {\textbf{Thesis:} Synchronizing Real-Time Hardware Control Across TSN Capable Networks} 
    % \resumeItemListEnd
  \resumeEntryEnd
  % \vspace{-15pt}
  \resumeEntryStart
    \resumeEntryTSDL
      {Mechatronics Engineering \& Management B.Eng.}{Sep. 2014 -- Apr. 2020}
      {McMaster University}{Hamilton, ON}
  \resumeEntryEnd

%-------------------------------------------------- EXPERIENCE --------------------------------------------------

\section{}{Experience\hspace{473pt}}

  \resumeEntryStart
  \resumeEntryTSDL
    {Application Engineer}{Jan. 2023 -- Present}
    {Indie Semiconductor}{Toronto, ON}
  \resumeItemListStart
      \resumeItem {Designed and developed customized software and hardware solutions to meet customer requirements}
      \resumeItem {Implemented features within image sensor drivers of a camera vision processor SoC}
      \resumeItem {Developed tools for extracting, parsing and verification of embedded data from a video stream}
      \resumeItem {Created host-side tools for communicating with the camera vision processor SoC via I$^2$C}
      \resumeItem {Conducted functional safety and timing analyses to ensure industry standards compliance}
  \resumeItemListEnd
  \resumeEntryEnd

  \resumeEntryStart
  \resumeEntryTSDL
    {Graduate Research Assistant}{May 2020 -- Dec. 2020}
    {McMaster Automotive Resource Centre}{Hamilton, ON}
  \resumeItemListStart
      \resumeItem {Collaborated with NXP Semiconductor and an automotive OEM to migrate a motor control application to a centralized architecture}
      \resumeItem {Developed firmware for pre-production hardware (NXP S32S \& S32K3) configuring peripherals, clock trees and pin multiplexing}
      \resumeItem {Implemented time synchronization using Time Sensitive Networking (TSN) over Automotive Ethernet}
      \resumeItem {Utilized Lauterbach TRACE32 with JTAG debugging and ETM tracing for hardware and software testing and troubleshooting}
      \resumeItem {Conducted signal verification and timing analysis for networking and motor control applications}
  \resumeItemListEnd
  \resumeEntryEnd

  \resumeEntryStart
    \resumeEntryTSDL
      {Embedded Systems Specialist}{Oct. 2018 -- Aug. 2020}
      {NEUDOSE Satellite Team}{Hamilton, ON}
      \resumeItemListStart
      \resumeItem {Worked on a project funded by the Canadian Space Agency (CSA) to build and launch a satellite}
      \resumeItem {Developed CAN drivers for the satellite’s On-Board Computer (STM32) to support a network stack (CSP)}
      \resumeItem {Contributed to the development of the On-Board Computer's FreeRTOS based flight software in C/C++}
      \resumeItem {Designed a Printed Circuit Board using Altium Designer to serve as a prototype for testing flight software}
    \resumeItemListEnd
  \resumeEntryEnd

  \resumeEntryStart
  \resumeEntryTSDL
    {Research Assistant}{May 2017 -- Apr. 2020}
    {McMaster Centre for Software Certification}{Hamilton, ON}
    \resumeItemListStart
    \resumeItem {Developed a model-based Pacemaker, following Boston Scientific’s Pacemaker System Specification, using MATLAB Simulink on an embedded target NXP FRDM-K64F.}
    \resumeItem {Implemented UART communication in MATLAB Simulink to configure and monitor the Pacemaker in real-time using a Python graphical user interface (GUI).}
    \resumeItem {Created an automated hardware testing and verification process, including Arm Mbed based firmware in C++ and Python scripts for communication over UART.}
  \resumeItemListEnd
\resumeEntryEnd

  % \resumeEntryStart
  %   \resumeEntryTSDL
  %     {Undergraduate Research Assistant (Model Based Software)}{May 2017 -- April. 2020}
  %     {McMaster Centre for Software Certification}{Hamilton, ON}
  %   \resumeItemListStart
  %       \resumeItem {Developed a model based Pacemaker following Boston Scientific's Pacemaker System Specification using MATLAB Simulink running on an embedded target NXP FRDM-K64F}
  %       \resumeItem {Implemented UART communication in MATLAB Simulink configuring and monitoring the Pacemaker in real-time using a python graphical user interface (GUI)} 
  %       \resumeItem {Created an automated hardware testing and verification including Arm Mbed based firmware in C++ and python scripts communicating with the hardware over UART}
  %   \resumeItemListEnd
  % \resumeEntryEnd

  % \resumeEntryStart
  %   \resumeEntryTSDL
  %     {Instructional Assistant Intern (IAI)}{May 2018 -- April 2019}
  %     {McMaster University}{Hamilton, ON}
  %   \resumeItemListStart
  %   % \resumeItem {Presented Graphics Design (CAD) and Programming (Python) concepts to over 100 students weekly}
  %     \resumeItem {Developed Raspberry Pi I$^2$C drivers for an Inertial Measurement Unit (IMU) and a pulse oximeter}
  %     \resumeItem{Created a python testing tool and a Golang server providing students with feedback on python assignments}
  %     \resumeItem{Setup docker containers to automate the print submission and monitoring process on 3D printers}
  %   \resumeItemListEnd
  % \resumeEntryEnd


%-------------------------------------------------- PROJECTS --------------------------------------------------

\section{}{Projects\hspace{490pt}}

  \resumeEntryStart
  \resumeEntryTD
    {RETINA (Realtime Indoor Navigation Assistant)}{May 2020}
  \resumeItemListStart
    \resumeItem {A navigation system to assist people with visual impairment navigate buildings utilizing Ultra-Wide Band (UWB) technology with sub-meter precision}
    \resumeItem{Implemented BLE communication between the mobile app and Decawave DW1000 UWB transceivers to retrieve the user's real-time position and heading}
    \resumeItem{Integrated the mobile app with OpenStreetMap API for indoor maps as well as Nominatim for reverse geocoding and Valhalla for route generation}
  \resumeItemListEnd
  \resumeEntryEnd

  \resumeEntryStart
    \resumeEntryTD
      {Booky}{Jan. 2018}
    \resumeItemListStart
      \resumeItem {A Cross-Platform mobile app (iOS \& Android) that allows the user to search for books by using a picture of the book built using Google flutter}
      \resumeItem {Used Google Cloud services for image search as well retrieving information about the book of interest}
    \resumeItemListEnd
  \resumeEntryEnd

  \resumeEntryStart
    \resumeEntryTD
      {Sumobot Challenge}{Mar. 2018}
    \resumeItemListStart
      \resumeItem {Selected the components and built the electrical circuitry for the Sumobot}
      \resumeItem {Developed a C++ arduino project for motor control as well sensor sampling (line detection, ultrasound)}
    \resumeItemListEnd
  \resumeEntryEnd

\section{}{Publications\hspace{463pt}}
\resumeEntryStart
    \resumeEntryTD
      {Two Simulink Models with Requirements for a Simple Controller of a Pacemaker Device}{Sep. 2022}
      \resumeItemListStart
        \resumeItem{Accepted at the 9th International Workshop on Applied Verification of Continuous and Hybrid Systems}
      \resumeItemListEnd
      \resumeEntryEnd
%-------------------------------------------------- PROGRAMMING SKILLS --------------------------------------------------
\section{}{Skills\hspace{500pt}}
 \resumeEntryStart
  \resumeEntryS{Languages \hspace{26pt}} {C, Python, C++, ARM Assembly, JavaScript, Java, Dart, Verilog, SQL}
  \resumeEntryS{Tools \hspace{52pt}} {CMake, Ninja, GDB, OpenOCD, Git, Docker, SVN}
  \resumeEntryS{Software \hspace{34pt}} {MATLAB, Simulink, Altium Designer, Lauterbach TRACE32, STM32CubeMX}
  \resumeEntryS{Hardware \hspace{30pt}} {ARM Cortex-M (STM32, NXP S32K3), ARM Cortex-R (NXP S32S24), PowerPC (NXP MPC5777C), FPGA}
  \resumeEntryS{Communication } {CAN, Automotive Ethernet (TSN), UART, SPI, I$^2$C, MQTT, UDP, TCP/IP}
 \resumeEntryEnd